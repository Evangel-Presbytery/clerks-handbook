% Options for packages loaded elsewhere
\PassOptionsToPackage{unicode}{hyperref}
\PassOptionsToPackage{hyphens}{url}
%
\documentclass[
]{book}
\title{A Handbook for Evangel Presbytery Clerks}
\author{Office of the Stated Clerk}
\date{2022-02-03}

\usepackage{amsmath,amssymb}
\usepackage{lmodern}
\usepackage{iftex}
\ifPDFTeX
  \usepackage[T1]{fontenc}
  \usepackage[utf8]{inputenc}
  \usepackage{textcomp} % provide euro and other symbols
\else % if luatex or xetex
  \usepackage{unicode-math}
  \defaultfontfeatures{Scale=MatchLowercase}
  \defaultfontfeatures[\rmfamily]{Ligatures=TeX,Scale=1}
\fi
% Use upquote if available, for straight quotes in verbatim environments
\IfFileExists{upquote.sty}{\usepackage{upquote}}{}
\IfFileExists{microtype.sty}{% use microtype if available
  \usepackage[]{microtype}
  \UseMicrotypeSet[protrusion]{basicmath} % disable protrusion for tt fonts
}{}
\makeatletter
\@ifundefined{KOMAClassName}{% if non-KOMA class
  \IfFileExists{parskip.sty}{%
    \usepackage{parskip}
  }{% else
    \setlength{\parindent}{0pt}
    \setlength{\parskip}{6pt plus 2pt minus 1pt}}
}{% if KOMA class
  \KOMAoptions{parskip=half}}
\makeatother
\usepackage{xcolor}
\IfFileExists{xurl.sty}{\usepackage{xurl}}{} % add URL line breaks if available
\IfFileExists{bookmark.sty}{\usepackage{bookmark}}{\usepackage{hyperref}}
\hypersetup{
  pdftitle={A Handbook for Evangel Presbytery Clerks},
  pdfauthor={Office of the Stated Clerk},
  hidelinks,
  pdfcreator={LaTeX via pandoc}}
\urlstyle{same} % disable monospaced font for URLs
\usepackage{longtable,booktabs,array}
\usepackage{calc} % for calculating minipage widths
% Correct order of tables after \paragraph or \subparagraph
\usepackage{etoolbox}
\makeatletter
\patchcmd\longtable{\par}{\if@noskipsec\mbox{}\fi\par}{}{}
\makeatother
% Allow footnotes in longtable head/foot
\IfFileExists{footnotehyper.sty}{\usepackage{footnotehyper}}{\usepackage{footnote}}
\makesavenoteenv{longtable}
\usepackage{graphicx}
\makeatletter
\def\maxwidth{\ifdim\Gin@nat@width>\linewidth\linewidth\else\Gin@nat@width\fi}
\def\maxheight{\ifdim\Gin@nat@height>\textheight\textheight\else\Gin@nat@height\fi}
\makeatother
% Scale images if necessary, so that they will not overflow the page
% margins by default, and it is still possible to overwrite the defaults
% using explicit options in \includegraphics[width, height, ...]{}
\setkeys{Gin}{width=\maxwidth,height=\maxheight,keepaspectratio}
% Set default figure placement to htbp
\makeatletter
\def\fps@figure{htbp}
\makeatother
\setlength{\emergencystretch}{3em} % prevent overfull lines
\providecommand{\tightlist}{%
  \setlength{\itemsep}{0pt}\setlength{\parskip}{0pt}}
\setcounter{secnumdepth}{5}
\usepackage{booktabs}
\usepackage{amsthm}
\makeatletter
\def\thm@space@setup{%
  \thm@preskip=8pt plus 2pt minus 4pt
  \thm@postskip=\thm@preskip
}
\makeatother
\ifLuaTeX
  \usepackage{selnolig}  % disable illegal ligatures
\fi
\usepackage[]{natbib}
\bibliographystyle{plainnat}

\begin{document}
\maketitle

{
\setcounter{tocdepth}{1}
\tableofcontents
}
\hypertarget{introduction}{%
\chapter{Introduction}\label{introduction}}

The introduction, if there is any, should be put here. This is what people will see first when they land on this book.

\hypertarget{overview}{%
\chapter{Overview}\label{overview}}

\textbf{Assignment: Alex}

\hypertarget{helpful-resources}{%
\chapter{Helpful Resources}\label{helpful-resources}}

\textbf{Assignment: Daniel}

Helpful!

\hypertarget{keeping-minutes}{%
\chapter{Keeping Minutes}\label{keeping-minutes}}

\textbf{Assignment: Alex}

\emph{Notes from Lucas that you may consider using:}

Meeting minutes are very important. The are the official record of the decisions of a deliberative body. Once approved, they are the final word on what happened and what was decided at a meeting.

A church has many committees that meet regularly, and many of those committees will choose to take minutes. It's not necessary for a clerk of session to have a permanent record of all those minutes. Instead, a clerk of session should be sure to have a permanent record of both regular and specially-called meeting minutes of the following groups:

\begin{itemize}
\tightlist
\item
  Session Meeting Minutes
\item
  Deacon Meeting Minutes
\item
  Congregational Meeting Minutes
\end{itemize}

You can find some helpful suggestions on how to organize your minutes \href{organization-of-files.html}{here}.

\begin{center}\rule{0.5\linewidth}{0.5pt}\end{center}

\href{https://docs.google.com/document/d/1WZ4PrWPCNPNsUEmfXLPNXCNJz5qaB09T3HXnCkPh2ak/edit?usp=sharing}{From Evangel Presbytery Bylaws}

APPENDIX 1: GUIDELINES FOR KEEPING SESSION MINUTES

A. Session Minutes must include the following:

A statement of the date, time, place, and purpose of the meeting (Stated, Called, adjourned stated, etc.)
If the meeting is a called meeting the minutes must include the portion of the call that indicates the purpose of the meeting.
That the meeting was opened and closed with prayer.
The names of all in attendance or absent from the meeting.
That a quorum was present for the meeting.
Communications received, and any action taken because of the communications.
Approval of minutes from previous meetings.
A statement that indicates the review and approval of the Diaconate minutes.
A statement that indicates the receiving of the Treasurer's report as information.
The actions of the Session including all the motions adopted and business transacted.

B. When applicable, the minutes shall include the following:

A record of all covenant baptisms, baptisms upon profession of faith, and dates of the celebration of the Lord's Supper.
Election of commissioners to presbytery.
A record of the call and purpose for each congregational or corporation meeting.
A motion calling for the nomination and election of church officers.
A record of the officers nominated, trained and examined by the Session.
The December or the following January minutes must include a statement that records the pastor's annual housing allowance approved by the Session.
The BCO requires Sessions to ``\ldots keep a fair record of baptisms, of those admitted to the Lord's table, of non-communing members, and of the births, marriages, deaths, and dismissions of church members and of any baptisms and marriage ceremonies presided over by the pastors of the church outside its bounds'' (BCO 14.8).
A copy of the approved operating budget and the yearly statistical report should be attached to December's minutes.

All minutes should be typed. They should be simple in word and in formatting. Email should be used when circulating minutes among members of the Session. The minutes should be included in the body of the email and not as an attachment. Once minutes have been officially adopted by the Session, they should be archived digitally using the PDF file format. To archive newly approved minutes, they should be added to the existing PDF archive. The pages should be consecutively numbered, leaving no blank pages between meetings and no records left out. As with all important digital documents, great care should be taken to ensure that Session minute archives have both local and remote backups. Sessions may keep a physical hard copy of their minutes in addition to their digital copy. If they choose to do so, however, they must ensure that the physical archives do not conflict with the digital ones.

C. After the Meeting:

Complete any necessary correspondence as required by actions taken by the Session.
Send transfer of the communicant's membership certificates to other churches.
Update the church membership records as required by action of the Session (Additions, removals, etc.).

\hypertarget{church-membership}{%
\chapter{Church Membership}\label{church-membership}}

\textbf{Assignment: Josh}

yo

\hypertarget{church-discipline}{%
\chapter{Church Discipline}\label{church-discipline}}

The Evangel Presbytery Book of Church Order addresses church discipline in chapters 30-49 (see \href{https://evangel.pressbooks.com/chapter/30-discipline-its-nature-subjects-and-ends/}{here}). This chapter exists to be a help and a short-hand reference for clerks where \emph{judicial process} is necessary (see \href{https://evangel.pressbooks.com/chapter/30-discipline-its-nature-subjects-and-ends/}{BCO 30.1}). It pertains, in other words, to church discipline that is \emph{formal} rather than \emph{informal}.

Clerks have specific duties in cases of discipline, and those duties are outlined below. But aside from their specific duties, clerks will often be called upon to help others understand disciplinary procedure. It is essential, therefore, for clerks to understand both the theoretical and practical aspects of church discipline, and there is no substitute for carefully reading the entire section on church discipline in our BCO. This page is only meant to be a help, and the BCO always takes precedence.

\hypertarget{general-principles}{%
\section{General Principles}\label{general-principles}}

\hypertarget{jurisdiction}{%
\subsection{Jurisdiction}\label{jurisdiction}}

Original jurisdiction in relation to Ministers of the Gospel pertains exclusively to the Presbytery, and in relation to other church members to the Session, unless the Session is unable to try the person or persons accused, in which case the Presbytery has the right of jurisdiction.

\hypertarget{offenses}{%
\subsection{Offenses}\label{offenses}}

An \href{https://evangel.pressbooks.com/chapter/32-offenses/}{offense}, the proper object of judicial process, is anything in the principles or practice of a church member professing faith in Christ, which is contrary to the Word of God. Offenses are either personal or general, private or public.

\hypertarget{church-censures}{%
\subsection{Church Censures}\label{church-censures}}

The \href{https://evangel.pressbooks.com/chapter/33-church-censures/}{censures} which may be inflicted by church courts are:

\begin{itemize}
\tightlist
\item
  admonition,
\item
  suspension from the Sacraments,
\item
  suspension from office,
\item
  deposition from office, and
\item
  excommunication.
\end{itemize}

\hypertarget{the-parties-in-cases-of-process}{%
\subsection{The Parties in Cases of Process}\label{the-parties-in-cases-of-process}}

The original and only parties in a case of process are the accuser and the accused. The accuser is always Evangel Presbytery, whose honor and purity are to be maintained.

Every indictment shall begin: ``In the name of Evangel Presbytery'' and shall conclude, ``against the peace, unity and purity of the Church, and the honor and majesty of the Lord Jesus Christ as the King and Head thereof.'' In every case the Church is the injured and accusing party, against the accused.

\hypertarget{duties-of-the-clerk}{%
\subsection{Duties of the Clerk}\label{duties-of-the-clerk}}

Overall, the Clerk's responsibility is to assemble the \textbf{Record of the Case} without delay. That record should include:

\begin{itemize}
\tightlist
\item
  the charges,
\item
  the answer,
\item
  the citations and returns thereto, and
\item
  the minutes herein required to be kept.
\end{itemize}

\hypertarget{specific-steps-in-judicial-process}{%
\section{Specific Steps in Judicial Process}\label{specific-steps-in-judicial-process}}

All charges brought before either the Session or Presbytery must be reduced to writing.

It is common in judicial proceedings for the session to appoint a Judicial Committee whose duty will be to digest and arrange all the papers, and to prescribe, under the direction of the court, the whole order of the proceedings. The clerk should work closely with the Judicial Committee to ensure that all papers and correspondence pertaining to the proceedings are recorded and stored properly.

\hypertarget{the-first-meeting}{%
\subsection{The First Meeting}\label{the-first-meeting}}

Nothing should be done at the first meeting of the court, unless by consent of parties, except to:

\begin{itemize}
\tightlist
\item
  appoint a prosecutor,
\item
  write up the indictment,
\item
  create a list of witnesses who support the indictment,
\item
  send the indictment, and the list of supporting witnesses, to the accused and cite all parties and their witnesses to appear and be heard at another meeting, which shall not be sooner than ten days after such citation.
\end{itemize}

Indictments and citations shall be delivered in person or in another manner providing verification of the date of receipt. In drawing the indictment, the times, places, and circumstances should, if possible, be particularly stated, that the accused may have an opportunity to make his defense.

\hypertarget{second-meeting}{%
\subsection{Second Meeting}\label{second-meeting}}

At the second meeting of the court:

\begin{itemize}
\tightlist
\item
  the charges shall be read to the accused, if present, and he shall be called upon to say whether he be guilty or not. If he confess, the court may deal with him according to its discretion; if he plead and take issue, the trial shall be scheduled and all parties and their witnesses cited to appear.
\item
  The trial shall not be sooner than fourteen (14) days after such citation.
\item
  Accused parties may plead in writing when they cannot be personally present.
\item
  Parties necessarily absent should have counsel assigned to them.
\end{itemize}

\hypertarget{the-trial}{%
\subsection{The Trial}\label{the-trial}}

The following order shall be observed at any trial:

\begin{enumerate}
\def\labelenumi{\arabic{enumi}.}
\tightlist
\item
  The Moderator shall charge the court.
\item
  The indictment shall be read, and the answer of the accused heard.
\item
  The witnesses for the prosecutor and then those for the accused shall be examined.
\item
  The parties shall be heard; first, the prosecutor, and then the accused, and the prosecutor shall close.
\item
  The roll shall be called, and the members may express their opinion in the cause.
\item
  The vote shall be taken, the verdict announced and judgment entered on the records.
\end{enumerate}

\textbf{The clerk must keep a record of the minutes of the trial.} Those minutes should include:

\begin{itemize}
\tightlist
\item
  the charges,
\item
  the answer,
\item
  all the testimony,
\item
  all such acts, orders, and decisions of the court relating to the case, as either party may desire, and
\item
  the judgment.
\end{itemize}

\hypertarget{appeals}{%
\subsection{Appeals}\label{appeals}}

An appeal is the transfer to a higher court of a judicial case on which judgment has been rendered in a lower court, and is allowable only to the party against whom the decision has been rendered.

An appeal cannot be made to any court other than the next higher, except with its consent.

Only those who have submitted to a regular trial are entitled to an appeal. Those who have not submitted to a regular trial are not entitled to an appeal.

The grounds of appeal are such as the following:

\begin{itemize}
\tightlist
\item
  any irregularity in the proceedings of the lower court;
\item
  refusal of reasonable indulgence to a party on trial;
\item
  receiving improper, or declining to receive proper, evidence;
\item
  hurrying to a decision before all the testimony is taken;
\item
  manifestation of prejudice in the case; and
\item
  mistake or injustice in the judgment and censure.
\end{itemize}

Written appeals must be filed with both the clerk of the lower court and the clerk of the higher court, \textbf{within thirty days of notification of the last court's decision}.

The clerk of the lower court must file a copy of all proceedings in connection with the case with the clerk of the higher court not more than thirty (30) days after receipt of notice of appeal.

\hypertarget{church-statistical-reports}{%
\chapter{Church Statistical Reports}\label{church-statistical-reports}}

\textbf{Assignment: Josh}

liars

\hypertarget{presbytery-records-review}{%
\chapter{Presbytery Records Review}\label{presbytery-records-review}}

\textbf{Assignment: Alex}

NOTE FROM LUCAS: I think church budgets should simply be added to this section, and removed as a stand-alone section. the only reason why clerks need to concern themselves with church budgets is so that they can submit them to the presbytery each year.

\hypertarget{organization-of-files}{%
\chapter{Organization of Files}\label{organization-of-files}}

Overall, it's a good idea to keep all important files together in one area.

\hypertarget{meeting-minutes}{%
\section{Meeting Minutes}\label{meeting-minutes}}

It's important to be consistent with your files names. We suggest the following:

YYYY.MM.DD - Church Name or Abbreviation \textbar{} Body and Meeting

So, for instance: ``2022.02.03 - TRC Session Meeting Minutes'' or ``2021.09.23 - TRC Specially Called Session Meeting Minutes''.

To organize your files, we suggest the following:

\begin{itemize}
\tightlist
\item
  Meeting Minutes

  \begin{itemize}
  \tightlist
  \item
    Session Minutes

    \begin{itemize}
    \tightlist
    \item
      2020
    \item
      2019
    \item
      2018
    \end{itemize}
  \item
    Deacon Meeting Minutes
  \item
    Congregational Meeting Minutes
  \end{itemize}
\end{itemize}

To ensure that there is one and only one \emph{official} version of any minutes, we discourage including temporary folders such as ``Waiting for Session Approval'' in the file structure listed above for the permanent record.

If you follow our \href{keeping-minutes.html}{guidelines for keeping minutes}, then minutes which are in the process of being approved will be sent out by email. Everyone who had a part in the meeting will then have a time-stamped record of the minutes. Once final changes are made and minutes have been approved, however, they should be added to the permanent record.

\hypertarget{attachments}{%
\subsection{Attachments}\label{attachments}}

At times, attachments will need to be added to approved minutes. When added to the permanent record, those attachements should be included in the file containing the minutes and \emph{not} left as a separate file.

\hypertarget{trials}{%
\section{Trials}\label{trials}}

The overall duties of a clerk of session regarding official church discipline are described \href{church-discipline.html}{here}. As stated \href{church-discipline.html\#duties-of-the-clerk}{here}, the clerk must ensure that there is a permanent, official record of the following:

\begin{itemize}
\tightlist
\item
  the charges,
\item
  the answer,
\item
  the citations and responses to the citations, and
\item
  the minutes of the court proceedings
\end{itemize}

The minutes of a court proceeding should include:

\begin{itemize}
\tightlist
\item
  the charges,
\item
  the answer,
\item
  all the testimony,
\item
  all such acts, orders, and decisions of the court relating to the case, as either party may desire, and
\item
  the judgment.
\end{itemize}

\hypertarget{miscellaneous}{%
\section{Miscellaneous}\label{miscellaneous}}

A clerk of session should work closely with a member of the church staff to organize and maintain the following additional documents:

\begin{itemize}
\tightlist
\item
  Bylaws
\item
  Annual Clerks Reports
\item
  Church Position Papers
\item
  Membership Records and Forms
\item
  Legal and Tax Documents
\item
  Annual Budgets
\item
  Documents Submitted to Evangel Presbytery for Review
\item
  \emph{etc.}
\end{itemize}

\hypertarget{backups}{%
\section{Backups}\label{backups}}

Most minutes are kept digitally these days, and so it's important that you maintain backups of your important files. A good backup plan will ensure that there are always 3 copies of your files at all times.

For instance:

\begin{enumerate}
\def\labelenumi{\arabic{enumi}.}
\tightlist
\item
  Files are on your computer, and
\item
  Files are being synced regularly with a cloud service like \href{https://drive.google.com}{Google Drive} or \href{https://dropbox.com}{Dropbox}, and
\item
  Files are being backed up regularly with a service like \href{https://www.backblaze.com}{Backblaze}.
\end{enumerate}

For added security, you can also connect to a hard drive on your local network to run regular backups.

\end{document}
