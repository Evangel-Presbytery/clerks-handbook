% Options for packages loaded elsewhere
\PassOptionsToPackage{unicode}{hyperref}
\PassOptionsToPackage{hyphens}{url}
%
\documentclass[
]{book}
\usepackage{amsmath,amssymb}
\usepackage{lmodern}
\usepackage{iftex}
\ifPDFTeX
  \usepackage[T1]{fontenc}
  \usepackage[utf8]{inputenc}
  \usepackage{textcomp} % provide euro and other symbols
\else % if luatex or xetex
  \usepackage{unicode-math}
  \defaultfontfeatures{Scale=MatchLowercase}
  \defaultfontfeatures[\rmfamily]{Ligatures=TeX,Scale=1}
\fi
% Use upquote if available, for straight quotes in verbatim environments
\IfFileExists{upquote.sty}{\usepackage{upquote}}{}
\IfFileExists{microtype.sty}{% use microtype if available
  \usepackage[]{microtype}
  \UseMicrotypeSet[protrusion]{basicmath} % disable protrusion for tt fonts
}{}
\makeatletter
\@ifundefined{KOMAClassName}{% if non-KOMA class
  \IfFileExists{parskip.sty}{%
    \usepackage{parskip}
  }{% else
    \setlength{\parindent}{0pt}
    \setlength{\parskip}{6pt plus 2pt minus 1pt}}
}{% if KOMA class
  \KOMAoptions{parskip=half}}
\makeatother
\usepackage{xcolor}
\IfFileExists{xurl.sty}{\usepackage{xurl}}{} % add URL line breaks if available
\IfFileExists{bookmark.sty}{\usepackage{bookmark}}{\usepackage{hyperref}}
\hypersetup{
  pdftitle={Handbook for Evangel Presbytery Clerks},
  pdfauthor={Office of the Stated Clerk of Evangel Presbytery},
  hidelinks,
  pdfcreator={LaTeX via pandoc}}
\urlstyle{same} % disable monospaced font for URLs
\usepackage{longtable,booktabs,array}
\usepackage{calc} % for calculating minipage widths
% Correct order of tables after \paragraph or \subparagraph
\usepackage{etoolbox}
\makeatletter
\patchcmd\longtable{\par}{\if@noskipsec\mbox{}\fi\par}{}{}
\makeatother
% Allow footnotes in longtable head/foot
\IfFileExists{footnotehyper.sty}{\usepackage{footnotehyper}}{\usepackage{footnote}}
\makesavenoteenv{longtable}
\usepackage{graphicx}
\makeatletter
\def\maxwidth{\ifdim\Gin@nat@width>\linewidth\linewidth\else\Gin@nat@width\fi}
\def\maxheight{\ifdim\Gin@nat@height>\textheight\textheight\else\Gin@nat@height\fi}
\makeatother
% Scale images if necessary, so that they will not overflow the page
% margins by default, and it is still possible to overwrite the defaults
% using explicit options in \includegraphics[width, height, ...]{}
\setkeys{Gin}{width=\maxwidth,height=\maxheight,keepaspectratio}
% Set default figure placement to htbp
\makeatletter
\def\fps@figure{htbp}
\makeatother
\setlength{\emergencystretch}{3em} % prevent overfull lines
\providecommand{\tightlist}{%
  \setlength{\itemsep}{0pt}\setlength{\parskip}{0pt}}
\setcounter{secnumdepth}{5}
\usepackage{booktabs}
\usepackage{amsthm}
\makeatletter
\def\thm@space@setup{%
  \thm@preskip=8pt plus 2pt minus 4pt
  \thm@postskip=\thm@preskip
}
\makeatother
\ifLuaTeX
  \usepackage{selnolig}  % disable illegal ligatures
\fi
\usepackage[]{natbib}
\bibliographystyle{plainnat}

\title{Handbook for Evangel Presbytery Clerks}
\author{Office of the Stated Clerk of Evangel Presbytery}
\date{2022-03-02}

\begin{document}
\maketitle

{
\setcounter{tocdepth}{1}
\tableofcontents
}
\hypertarget{overview}{%
\chapter{Overview}\label{overview}}

Welcome to Evangel Presbytery's handbook for clerks of session. This resource is not part of Evangel's officially adopted governing documents; it is simply intended to make the work of clerks of session easier, more efficient, and more reliable. It consolidates information from Evangel's \href{https://evangel.pressbooks.com/}{Book of Church Order} (BCO) and \href{https://evangelpresbytery.com/documents/}{Bylaws} as they pertain specifically to clerks of session. Additionally, this handbook offers practical resources and guidance for carrying out the clerk's duties.

\hypertarget{the-importance-of-keeping-good-records}{%
\section{The Importance of Keeping Good Records}\label{the-importance-of-keeping-good-records}}

The officers of Christ's church are called by the Lord to carry out justice among His people. Central to this work is the keeping of faithful records. The prophet Isaiah gives us a glimpse of the importance of this in the church:

\begin{quote}
Woe to those who enact evil statutes\\
And to those who constantly record unjust decisions,\\
So as to deprive the needy of justice\\
And rob the poor of My people of their rights,\\
So that widows may be their spoil\\
And that they may plunder the orphans.
\end{quote}

God cares about the records we keep. The recording of truth helps ensure that the poor and the needy are well cared for in the church.

\hypertarget{the-stated-clerk-defined}{%
\section{The Stated Clerk Defined}\label{the-stated-clerk-defined}}

The BCO organizes Evangel Presbytery's government into two types of ``court,'' church sessions and presbyteries (\href{https://evangel.pressbooks.com/chapter/12-church-courts-in-general/}{BCO 12.1--2}). Each of these courts must have a stated clerk who is responsible for its records. The presbytery as a whole has a stated clerk, and each church session is responsible for electing its own stated clerk (\href{https://evangel.pressbooks.com/chapter/6-the-organization-of-a-particular-church/}{BCO 6.3}).

``Stated clerk'' is a general term for the officially elected clerk of any court in the presbytery. When he is elected by a session to keep the records of that session, he is often referred to as the ``clerk of session.'' As such, he is responsible for keeping, organizing, understanding, and furnishing the church's records:

\begin{quote}
It is the duty of the Clerk, besides recording the transactions, to preserve the records carefully, and to grant extracts from them whenever properly required. Such extracts, under the hand of the Clerk, shall be evidence to any ecclesiastical court, and to every part of the Church. (\href{https://evangel.pressbooks.com/chapter/12-church-courts-in-general/}{BCO 12.4})
\end{quote}

\hypertarget{areas-of-responsibility}{%
\section{Areas of Responsibility}\label{areas-of-responsibility}}

This handbook is organized according to the stated clerk's areas of responsibility. These typically include (but are not necessarily limited to) the following:

\begin{itemize}
\tightlist
\item
  \protect\hyperlink{governing-documents}{Governing Documents}
\item
  \protect\hyperlink{keeping-minutes}{Keeping Minutes}
\item
  \protect\hyperlink{membership-records}{Membership Records}
\item
  \protect\hyperlink{church-discipline-records}{Church Discipline Records}
\item
  \protect\hyperlink{records-organization}{Records Organization}
\item
  \protect\hyperlink{submission-of-records-to-presbytery}{Submission of Records to Presbytery}
\end{itemize}

\hypertarget{the-clerks-help}{%
\section{The Clerk's Help}\label{the-clerks-help}}

It is common for the stated clerk to delegate some of his duties to other individuals, especially in the context of a larger church or session where the keeping of records is more complex. In particular, the task of keeping minutes is often delegated to a \emph{recording clerk}, or \emph{clerk pro tempore} (\emph{clerk pro tem} for short), as it can be difficult to participate in a meeting while also recording the meeting's proceedings.

The clerk will also often utilize other individuals in keeping records for committees, subcommittees, or other boards of the church. For instance, the clerk should normally rely on the secretary of the board of deacons to keep good records for the deacons.

Even when help from others is utilized, the stated clerk is still finally responsible for compiling the records kept, and for ensuring their accuracy.

\hypertarget{governing-documents}{%
\chapter{Governing Documents}\label{governing-documents}}

Need to discuss:

\begin{itemize}
\tightlist
\item
  Civil documents
\item
  Church documents
\item
  Relationship between church bylaws and the BCO

  \begin{itemize}
  \tightlist
  \item
    bylaws are in addition to the BCO and are subordinate to it.
  \item
    May be worth mentioning kinds of areas where a church may be more specific: voting age, officer terms, etc.
  \end{itemize}
\item
  Perhaps mention the clerk's responsibility to understand the process for updating any of these documents\ldots{}
\end{itemize}

\hypertarget{keeping-minutes}{%
\chapter{Keeping Minutes}\label{keeping-minutes}}

Evangel Presbytery's official standards for keeping minutes can be found in \href{https://docs.google.com/document/d/1WZ4PrWPCNPNsUEmfXLPNXCNJz5qaB09T3HXnCkPh2ak/edit?usp=sharing}{Appendix 1 of the presbytery's bylaws}. The information included in this section of the handbook is intended to supplement the presbytery's standards, without adding further requirements. \textbf{\emph{There is some helpful info here about what the BCO requires, however.}}

\hypertarget{overview-1}{%
\section{Overview}\label{overview-1}}

\hypertarget{the-importance-of-good-minutes}{%
\subsection{The Importance of Good Minutes}\label{the-importance-of-good-minutes}}

Meeting minutes are the official record of the decisions of a deliberative body. Once approved, they are the final word on what happened and what was decided at a meeting. The keeping of minutes is also typically the clerk's most consistent and frequent area of responsibility. In short, minutes must be kept with utmost care. Good minutes help promote integrity and accountability, and they act as a practical reference for future discussions and decisions of the session.

\hypertarget{principles-for-keeping-good-minutes}{%
\subsection{Principles for Keeping Good Minutes}\label{principles-for-keeping-good-minutes}}

\hypertarget{clarity}{%
\subsubsection{Clarity}\label{clarity}}

Put yourself in the shoes of a session member reading your minutes ten years from now. Ask yourself, Will he have enough information to clearly understand what decision was made, who was involved, and why the decision was made?

\hypertarget{concision}{%
\subsubsection{Concision}\label{concision}}

Minutes are intended to record essential information only. They should not to be a detailed play-by-play record of delibertaion. The most important thing to record in minutes is any official action taken by the session.

\hypertarget{consistency}{%
\subsubsection{Consistency}\label{consistency}}

There are many different ways to format and style spelling, punctuation, abbreviations, outline structure, etc. The most important thing is to be consistent in how you handle these matters of formatting and style. Consistency helps minutes to be more clear, and it also makes them easier to search for names and other information.

\hypertarget{what-minutes-should-be-kept}{%
\section{What Minutes Should Be Kept?}\label{what-minutes-should-be-kept}}

A clerk of session should be sure to have a permanent record of both regular and special called meetings of the following groups:

\begin{itemize}
\tightlist
\item
  The session, a.k.a. the board of elders.
\item
  Session-appointed commissions: ``A Commission is authorized to deliberate upon and conclude the business referred to it. It shall keep a full record of its proceedings, which shall be submitted to the court appointing it, entered on its minutes, and regarded and treated as the action of the court'' (\href{https://evangel.pressbooks.com/chapter/17-ecclesiastical-commissions/}{BCO 17.1}).
\item
  The board of deacons. Deacons' minutes do not need to be submitted to the presbytery for review, but they do need to be submitted to the session. The clerk of session will most often rely on the secretary of the board of deacons to keep minutes for deacons meetings, but it is still the clerk's responsibility to ensure these minutes are accurate, well organized, and available for review.
\item
  The congregation. There are some special considerations for keeping good congregational meeting minutes, as the size of the body meeting can make proceedings more complicated.
\end{itemize}

A church has many committees that meet regularly, and many of those committees will choose to take minutes. However, it is not necessary for a clerk of session to have a permanent record of all those minutes.

\hypertarget{adoption-process-for-session-minutes}{%
\section{Adoption Process for Session Minutes}\label{adoption-process-for-session-minutes}}

\begin{itemize}
\tightlist
\item
  After minutes are taken, send minutes to moderator for review and correction.
\item
  Send minutes to session for review before the meeting at which they will be adopted.
\item
  File minutes\ldots link
\end{itemize}

\hypertarget{examples}{%
\section{Examples}\label{examples}}

\hypertarget{basic-minutes-template}{%
\subsection{Basic Minutes Template}\label{basic-minutes-template}}

We recommend clerks come to each meeting with a basic minutes outline already prepared.\ldots{}

\hypertarget{sample-set-of-session-meeting-minutes-download}{%
\subsection{Sample Set of Session Meeting Minutes (download)}\label{sample-set-of-session-meeting-minutes-download}}

\hypertarget{sample-set-of-congregational-meeting-minutes-download}{%
\subsection{Sample Set of Congregational Meeting Minutes (download)}\label{sample-set-of-congregational-meeting-minutes-download}}

\hypertarget{organizing-your-sessions-minutes}{%
\section{Organizing Your Session's Minutes}\label{organizing-your-sessions-minutes}}

You can find some helpful suggestions on how to organize your minutes {[}here{]}{[}Records-Organization{]}.

\hypertarget{optional-guidelines}{%
\section{Optional Guidelines}\label{optional-guidelines}}

For clerks who want to follow a more thorough set of standards for session minutes, we've provided the set of guidelines below. We note, however, that these guidelines are \emph{not} required by the presbytery. They are simply offered as an optional aid for ensuring clarity, concision, and consistency.

\begin{enumerate}
\def\labelenumi{\arabic{enumi}.}
\item
  Write all minutes in the body of an email, rather than in Microsoft Word or another word processing application. Email programs allow basic text formatting (bold, italics) without the problems associated with attachments. The principle is to avoid attachments because they clutter up hard drives and can't be searched as easily as email may be. So keep your minutes simple in wording and formatting, and always write and circulate them in the body of an email.
\item
  At the beginning of each set of minutes, place a header on the left side, in the following format:
\end{enumerate}

Name of Institution
Type of Meeting (including the specific body meeting)
Time, Month, Date, YYYY
Location of meeting (or name of software used if meeting was held virtually, e.g., ``via Zoom'')

For a regular session meeting at Trinity Reformed Church, the header would be as follows:

Trinity Reformed Church, Bloomington, IN
Regular Scheduled Session Meeting
7:00pm, October 11, 2012
at Trinity Reformed Church

\begin{enumerate}
\def\labelenumi{\arabic{enumi}.}
\setcounter{enumi}{2}
\tightlist
\item
  Below the header (one line intervening), type ``Present:'', followed by a list of all individuals in attendance at some point during the meeting. After each individual, put in parentheses his position/function in relation to the body meeting; for men with the same position, this may be put after all their names. Separate entries with a semicolon where it separates distinct positions; use a comma between names of the same position. Note these practices in the following example:
\end{enumerate}

Present: John Owen (moderator); John Bunyan, Richard Baxter (associate pastors); Jeremy Taylor, Richard Sibbes, Jeremiah Burroughs (assistant pastors); Martin Luther, John Calvin, William Tyndale (elders); Philip Melanchthon (chairman, deacons board); Theodore Beza, Zacharias Ursinus, Jan Hus (deacons); Joshua Congrove (recording clerk); Thomas Aquinas, Clement of Rome (guests of session)

\begin{enumerate}
\def\labelenumi{\arabic{enumi}.}
\setcounter{enumi}{3}
\item
  Below the previous list, type ``Excused Absences:'', followed by a list of all regular session members whose absences were excused by the moderator. If desired, type another line beginning with ``Other Absences:'' in which you list all absences not explicitly excused.
\item
  Below the prefatory material, and before enumeration of each item, place a paragraph summarizing when the meeting began, as well as the presenter, Scripture reference, and theme of the devotions. No summary of the devotions is needed. Then indicate who opened the meeting in prayer.
\item
  Minutes should be numbered according to topic. Sometimes, when multiple discussions or decisions happen on one topic, it's best to use a general heading (e.g., ``Baptism issues'' or ``Financial matters''), and then specific subheadings to organize the information. Other times, simply provide a sentence or two summarizing the discussion, and then place any action items beneath the relevant section.
\end{enumerate}

Note: When the minutes are compiled and revised, they should be grouped according to their proper topic, even if this does not match the chronology in which they occurred. For example, if discussion in the early part of the meeting concerned the financial matters of Bloomington Christian Schoolhouse, and discussion later on dealt with appointments to the Schoolhouse board, these notes should be combined under an appropriate heading. Rarely, it's useful to keep the notes separate if preserving the chronological order in which they were discussed is absolutely essential; but usually, it's least confusing to gather similar elements together.

\begin{enumerate}
\def\labelenumi{\arabic{enumi}.}
\setcounter{enumi}{6}
\tightlist
\item
  The cardinal rule of taking minutes is to write enough, but not more. Crucial to include are action items (see below), motions, and resolutions. Helpful to include are sentences summarizing reports made by members (e.g., the financial report), and a brief (one- or two-sentence) statement of what was discussed. Since minutes are properly intended to record only official actions of the board, please do not include opinions or consensus of the board or its individuals. The exception to this rule is when a member of the board would like his opposition to the session's decision recorded in the minutes as a protest. In this case, however, he then needs to inform the entire board of his wish to record his opposition, and the moderator will help the recorder of the minutes to handle it properly.
\end{enumerate}

In addition, many decisions will be declared by the moderator to be ``by consensus.'' Such decisions are to be recorded as official actions of the board, and should be prefaced ``By consensus'' (see below). Unless so directed, do not write down the details of the discussion, nor the individual points made in the process. Record where the board finally arrives, not how they got there.

\begin{enumerate}
\def\labelenumi{\arabic{enumi}.}
\setcounter{enumi}{7}
\item
  Below the relevant topic, indicate tasks assigned to individuals in the following format:
  ---NEW ACTION ITEM: Joe Schmoe will write a valediction to his name (copying elder Bob Smith on it), expressing his joy at being able to acquire a new designation.
\item
  Record the names of anyone who takes significant action, to include motions, of course, but also action such as opening or closing in prayer. Often, if the action taken is informal or unanimous, you may also use ``M/S/A'' or ``by consensus'', depending on the nature of the action (and on whether you're able to record the names fast enough!)
\item
  Votes in the negative do not need to be recorded unless requested by the voter, or unless the moderator so directs.
\item
  When recording names of individuals, always give the first and last name, unless that individual has been mentioned in a previous sentence of the same numbered item. (For Jürgen von Hagen, spell his name thus.)
\item
  Do not hesitate to remind the moderator of his various duties if he overlooks them, particularly if they bear directly upon the minutes. Such duties include prayer at the open and close of the meeting, assigning action items to specific members of the board, ensuring that initial motions are given and seconded before proceeding with a vote, setting the time and place of the next meeting and who will give devotions, and so on.
\item
  If there is any more than one instance of orders of the day, I sum the total number of minutes granted in aggregate, and express this in a single motion, e.g., ``M/S/A for orders of the day for 45 minutes, in totality.''
\item
  Supplemental documents: Documents that are approved by the board should only be referenced within the minutes themselves, and then appended inline below the body of the minutes. If documents are especially long or dependent upon formatting that may be lost in email form, you may also attach the original document in .pdf form; regardless, documents must also be included inline.
\item
  Abbreviations for organizations or ministries of the church should only be used when clear and stated in full prior to their use. For example: The session received a report on New Geneva Academy (NGA) from pastor Smith. He explained that NGA is working on\ldots{}
\item
  Style: In general, our standard is Chicago Manual of Style. For commas, use the Oxford comma (i.e., including a comma before the final item in a list). Use only one space between sentences.
\end{enumerate}

Be scrupulous in your sentence structure, vocabulary, and style. Your goal is to produce a document in which the moderator will find nothing needing change. Likely you'll never achieve that, but it's something to aim for!

\begin{enumerate}
\def\labelenumi{\arabic{enumi}.}
\setcounter{enumi}{16}
\item
  Concluding matter: Always include a statement about when and where the next meeting will be, and who will provide devotions. Then indicate when the meeting was adjourned, and by whose motion and prayer.
\item
  Below the body of the minutes, but before any supplemental documents, write the closing signature as follows:
\end{enumerate}

Respectfully submitted,

Your Name, Title
Name of Body Meeting

Separate the minutes proper from any supplemental appended documents with a row of asterisks or hyphens.

\hypertarget{membership-records}{%
\chapter{Membership Records}\label{membership-records}}

\textbf{Assignment: Josh}

NOTE: Include annual statistical reports in this section.

\hypertarget{church-discipline-records}{%
\chapter{Church Discipline Records}\label{church-discipline-records}}

The Evangel Presbytery Book of Church Order addresses church discipline in chapters 30-49 (see \href{https://evangel.pressbooks.com/chapter/30-discipline-its-nature-subjects-and-ends/}{here}). This chapter exists to be a help and a short-hand reference for clerks where \emph{judicial process} is necessary (see \href{https://evangel.pressbooks.com/chapter/30-discipline-its-nature-subjects-and-ends/}{BCO 30.1}). It pertains, in other words, to church discipline that is \emph{formal} rather than \emph{informal}.

Clerks have specific duties in cases of discipline, and those duties are outlined below. But aside from their specific duties, clerks will often be called upon to help others understand disciplinary procedure. It is essential, therefore, for clerks to understand both the theoretical and practical aspects of church discipline, and there is no substitute for carefully reading the entire section on church discipline in our BCO. This page is only meant to be a help, and the BCO always takes precedence.

\hypertarget{general-principles}{%
\section{General Principles}\label{general-principles}}

\hypertarget{jurisdiction}{%
\subsection{Jurisdiction}\label{jurisdiction}}

Original jurisdiction in relation to Ministers of the Gospel pertains exclusively to the Presbytery, and in relation to other church members to the Session, unless the Session is unable to try the person or persons accused, in which case the Presbytery has the right of jurisdiction.

\hypertarget{offenses}{%
\subsection{Offenses}\label{offenses}}

An \href{https://evangel.pressbooks.com/chapter/32-offenses/}{offense}, the proper object of judicial process, is anything in the principles or practice of a church member professing faith in Christ, which is contrary to the Word of God. Offenses are either personal or general, private or public.

\hypertarget{church-censures}{%
\subsection{Church Censures}\label{church-censures}}

The \href{https://evangel.pressbooks.com/chapter/33-church-censures/}{censures} which may be inflicted by church courts are:

\begin{itemize}
\tightlist
\item
  admonition,
\item
  suspension from the Sacraments,
\item
  suspension from office,
\item
  deposition from office, and
\item
  excommunication.
\end{itemize}

\hypertarget{the-parties-in-cases-of-process}{%
\subsection{The Parties in Cases of Process}\label{the-parties-in-cases-of-process}}

The original and only parties in a case of process are the accuser and the accused. The accuser is always Evangel Presbytery, whose honor and purity are to be maintained.

Every indictment shall begin: ``In the name of Evangel Presbytery'' and shall conclude, ``against the peace, unity and purity of the Church, and the honor and majesty of the Lord Jesus Christ as the King and Head thereof.'' In every case the Church is the injured and accusing party, against the accused.

\hypertarget{duties-of-the-clerk}{%
\subsection{Duties of the Clerk}\label{duties-of-the-clerk}}

Overall, the Clerk's responsibility is to assemble the \textbf{Record of the Case} without delay. That record should include:

\begin{itemize}
\tightlist
\item
  the charges,
\item
  the answer,
\item
  the citations and returns thereto, and
\item
  the minutes herein required to be kept.
\end{itemize}

\hypertarget{specific-steps-in-judicial-process}{%
\section{Specific Steps in Judicial Process}\label{specific-steps-in-judicial-process}}

All charges brought before either the Session or Presbytery must be reduced to writing.

It is common in judicial proceedings for the session to appoint a Judicial Committee whose duty will be to digest and arrange all the papers, and to prescribe, under the direction of the court, the whole order of the proceedings. The clerk should work closely with the Judicial Committee to ensure that all papers and correspondence pertaining to the proceedings are recorded and stored properly.

\hypertarget{the-first-meeting}{%
\subsection{The First Meeting}\label{the-first-meeting}}

Nothing should be done at the first meeting of the court, unless by consent of parties, except to:

\begin{itemize}
\tightlist
\item
  appoint a prosecutor,
\item
  write up the indictment,
\item
  create a list of witnesses who support the indictment,
\item
  send the indictment, and the list of supporting witnesses, to the accused and cite all parties and their witnesses to appear and be heard at another meeting, which shall not be sooner than ten days after such citation.
\end{itemize}

Indictments and citations shall be delivered in person or in another manner providing verification of the date of receipt. In drawing the indictment, the times, places, and circumstances should, if possible, be particularly stated, that the accused may have an opportunity to make his defense.

\hypertarget{second-meeting}{%
\subsection{Second Meeting}\label{second-meeting}}

At the second meeting of the court:

\begin{itemize}
\tightlist
\item
  the charges shall be read to the accused, if present, and he shall be called upon to say whether he be guilty or not. If he confess, the court may deal with him according to its discretion; if he plead and take issue, the trial shall be scheduled and all parties and their witnesses cited to appear.
\item
  The trial shall not be sooner than fourteen (14) days after such citation.
\item
  Accused parties may plead in writing when they cannot be personally present.
\item
  Parties necessarily absent should have counsel assigned to them.
\end{itemize}

\hypertarget{the-trial}{%
\subsection{The Trial}\label{the-trial}}

The following order shall be observed at any trial:

\begin{enumerate}
\def\labelenumi{\arabic{enumi}.}
\tightlist
\item
  The Moderator shall charge the court.
\item
  The indictment shall be read, and the answer of the accused heard.
\item
  The witnesses for the prosecutor and then those for the accused shall be examined.
\item
  The parties shall be heard; first, the prosecutor, and then the accused, and the prosecutor shall close.
\item
  The roll shall be called, and the members may express their opinion in the cause.
\item
  The vote shall be taken, the verdict announced and judgment entered on the records.
\end{enumerate}

\textbf{The clerk must keep a record of the minutes of the trial.} Those minutes should include:

\begin{itemize}
\tightlist
\item
  the charges,
\item
  the answer,
\item
  all the testimony,
\item
  all such acts, orders, and decisions of the court relating to the case, as either party may desire, and
\item
  the judgment.
\end{itemize}

\hypertarget{appeals}{%
\subsection{Appeals}\label{appeals}}

An appeal is the transfer to a higher court of a judicial case on which judgment has been rendered in a lower court, and is allowable only to the party against whom the decision has been rendered.

An appeal cannot be made to any court other than the next higher, except with its consent.

Only those who have submitted to a regular trial are entitled to an appeal. Those who have not submitted to a regular trial are not entitled to an appeal.

The grounds of appeal are such as the following:

\begin{itemize}
\tightlist
\item
  any irregularity in the proceedings of the lower court;
\item
  refusal of reasonable indulgence to a party on trial;
\item
  receiving improper, or declining to receive proper, evidence;
\item
  hurrying to a decision before all the testimony is taken;
\item
  manifestation of prejudice in the case; and
\item
  mistake or injustice in the judgment and censure.
\end{itemize}

Written appeals must be filed with both the clerk of the lower court and the clerk of the higher court, \textbf{within thirty days of notification of the last court's decision}.

The clerk of the lower court must file a copy of all proceedings in connection with the case with the clerk of the higher court not more than thirty (30) days after receipt of notice of appeal.

\hypertarget{records-organization}{%
\chapter{Records Organization}\label{records-organization}}

Overall, it's a good idea to keep all important files together in one area.

\hypertarget{meeting-minutes}{%
\section{Meeting Minutes}\label{meeting-minutes}}

It's important to be consistent with your files names. We suggest the following:

YYYY.MM.DD - Church Name or Abbreviation \textbar{} Body and Meeting

So, for instance: ``2022.02.03 - TRC Session Meeting Minutes'' or ``2021.09.23 - TRC Specially Called Session Meeting Minutes''.

To organize your files, we suggest the following:

\begin{itemize}
\tightlist
\item
  Meeting Minutes

  \begin{itemize}
  \tightlist
  \item
    Session Minutes

    \begin{itemize}
    \tightlist
    \item
      2020
    \item
      2019
    \item
      2018
    \end{itemize}
  \item
    Deacon Meeting Minutes
  \item
    Congregational Meeting Minutes
  \end{itemize}
\end{itemize}

To ensure that there is one and only one \emph{official} version of any minutes, we discourage including temporary folders such as ``Waiting for Session Approval'' in the file structure listed above for the permanent record.

If you follow our \href{keeping-minutes.html}{guidelines for keeping minutes}, then minutes which are in the process of being approved will be sent out by email. Everyone who had a part in the meeting will then have a time-stamped record of the minutes. Once final changes are made and minutes have been approved, however, they should be added to the permanent record.

\hypertarget{attachments}{%
\subsection{Attachments}\label{attachments}}

At times, attachments will need to be added to approved minutes. When added to the permanent record, those attachements should be included in the file containing the minutes and \emph{not} left as a separate file.

\hypertarget{trials}{%
\section{Trials}\label{trials}}

The overall duties of a clerk of session regarding official church discipline are described \href{church-discipline.html}{here}. As stated \href{church-discipline.html\#duties-of-the-clerk}{here}, the clerk must ensure that there is a permanent, official record of the following:

\begin{itemize}
\tightlist
\item
  the charges,
\item
  the answer,
\item
  the citations and responses to the citations, and
\item
  the minutes of the court proceedings
\end{itemize}

The minutes of a court proceeding should include:

\begin{itemize}
\tightlist
\item
  the charges,
\item
  the answer,
\item
  all the testimony,
\item
  all such acts, orders, and decisions of the court relating to the case, as either party may desire, and
\item
  the judgment.
\end{itemize}

\hypertarget{miscellaneous}{%
\section{Miscellaneous}\label{miscellaneous}}

A clerk of session should work closely with a member of the church staff to organize and maintain the following additional documents:

\begin{itemize}
\tightlist
\item
  Bylaws
\item
  Annual Clerks Reports
\item
  Church Budgets
\item
  Church Position Papers
\item
  Membership Records and Forms
\item
  Legal and Tax Documents
\item
  Annual Budgets
\item
  Documents Submitted to Evangel Presbytery for Review
\item
  \emph{etc.}
\end{itemize}

\hypertarget{backups}{%
\section{Backups}\label{backups}}

Most minutes are kept digitally these days, and so it's important that you maintain backups of your important files. A good backup plan will ensure that there are always 3 copies of your files at all times.

For instance:

\begin{enumerate}
\def\labelenumi{\arabic{enumi}.}
\tightlist
\item
  Files are on your computer, and
\item
  Files are being synced regularly with a cloud service like \href{https://drive.google.com}{Google Drive} or \href{https://dropbox.com}{Dropbox}, and
\item
  Files are being backed up regularly with a service like \href{https://www.backblaze.com}{Backblaze}.
\end{enumerate}

For added security, you can also connect to a hard drive on your local network to run regular backups.

\hypertarget{submission-of-records-to-presbytery}{%
\chapter{Submission of Records to Presbytery}\label{submission-of-records-to-presbytery}}

\textbf{Assignment: Alex}

NOTE FROM LUCAS: I think church budgets should simply be added to this section, and removed as a stand-alone section. the only reason why clerks need to concern themselves with church budgets is so that they can submit them to the presbytery each year.

\begin{itemize}
\tightlist
\item
  Notations
\item
  Exceptions of Form
\item
  Exceptions of Substance
\end{itemize}

Checklist for submitting minutes to presbytery

Link to bylaws appendix guidelines

\hypertarget{helpful-resources}{%
\chapter{Helpful Resources}\label{helpful-resources}}

\textbf{Assignment: Daniel}

Helpful!

\end{document}
